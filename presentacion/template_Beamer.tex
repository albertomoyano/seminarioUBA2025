\documentclass[14pt,aspectratio=169]{beamer}
\usepackage[utf8]{inputenc}
\usepackage[spanish]{babel}
\usetheme{Copenhagen}
\usecolortheme{beaver}
\usepackage{helvet}
%\usepackage{IBMPlex}
%\usepackage{titlesec}
%\usepackage{titletoc}

\title{Edición ramificada\\ -- la edición por venir --}
\subtitle{Presentación de gbTeXpublisher}
\author{Alberto Moyano}
\date{11 de marzo de 2024}
%\institute{Universidad de Buenos Aires}

\setbeamertemplate{headline}{}
\setbeamertemplate{navigation symbols}{}
\setbeamercovered{transparent}

\begin{document}

% genera el primer frame
\begin{frame}
\titlepage
\end{frame}

%\begin{frame}{Presentación}
%	gbTeXpulisher
%\end{frame}


%% Outline frame
%\begin{frame}
%	\tableofcontents
%\end{frame}

\begin{frame}{Estado del arte}
	\begin{enumerate}
		\item Edición cíclica
		\item Edición ramificada
			\begin{enumerate}
				\item Modelo estrella
				\item Modelo árbol
			\end{enumerate}
	\end{enumerate}
\end{frame}


\begin{frame}{Un solo origen}
	Pero, ¿qué es la edición ramificada?

\end{frame}

\begin{frame}{Lenguajes de marcas}
	\begin{enumerate}
		\item Ligeros
			\begin{enumerate}
			\item Markdown
			\item Asciidoc
		\end{enumerate}
		\item Verboso
		\begin{enumerate}
			\item LaTeX
		\end{enumerate}
	\end{enumerate}
\end{frame}

\begin{frame}{gbTeXpublisher}
	Pequeña explicación del software
	\begin{enumerate}
	\item Ventajas
		\begin{enumerate}
		\item Ventaja 1
		\item Ventaja 2
		\end{enumerate}
	\end{enumerate}
	\begin{enumerate}
	\item Desventaja (para decir alguna)
		\begin{enumerate}
		\item Solo funciona en Linux
		\end{enumerate}
	\end{enumerate}
\end{frame}

\end{document}
