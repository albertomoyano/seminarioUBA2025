\documentclass{book}

% Deben cargarse temprano para parches de macros
\usepackage{etoolbox}
\usepackage{ifthen}

% Configuración geométrica detallada
\usepackage[paperwidth=155mm,
paperheight=230mm,
textwidth=110mm,
textheight=540pt,
centering,
includehead,
includefoot,
headsep=14pt,
top=35pt,
footskip=0mm,
footnotesep=14pt plus 0.1pt minus 0.1pt]{geometry}
% Definir el valor de página para este preámbulo
\newcommand{\valorEspecifico}{15.5x23cm}

%% Configuración de idioma
\usepackage[french,portuguese,italian,english,german,spanish,es-ucroman,es-noshorthands]{babel}
% Citas automáticas con estilo idiomático correcto
\usepackage[autostyle=true]{csquotes}
% Espaciado uniforme después de puntos
\frenchspacing
% Personalización del nombre del índice
\renewcommand{\spanishcontentsname}{Sumario}
% cambiamos el nombre de sigla
\newcommand{\acronymsname}{Índice de siglas}

% Soporte para fuentes OpenType
\usepackage{fontspec}
% Mejoras tipográficas micro
\usepackage[final]{microtype}
% Mejora espaciado sin modificar el interletrado
\microtypecontext{spacing=nonfrench}
%% solo trabajamos con protusion y sin expansion para Libertinus
\SetProtrusion
[ name=libertinus ]
{ encoding = * }
{ . = {50,50}, , = {40,40}, - = {30,30}, " = {50,50}, ( = {40,50}, ) = {50,40} }
\renewcommand{\normalsize}{\fontsize{10pt}{14pt}\selectfont}
\topskip=14pt

% Fuente principal Libertinus con características tipográficas
\setmainfont{Libertinus Serif}
[Numbers={OldStyle,Proportional},
Ligatures={TeX,Common,Discretionary},
RawFeature={+cv01,+cv02},% Variantes de caracteres opcionales
Scale=1.18]

% Matemáticas con Libertinus Math
\usepackage{unicode-math}
\setmathfont{Libertinus Math}[Scale=MatchLowercase]

%% Configuración de fuentes para chino
% \usepackage{luatexja-fontspec}
% \setmainjfont{FandolSong}

\setsansfont[Scale=MatchLowercase,
Ligatures=TeX,
Extension=.otf,
UprightFont=*-Regular,
ItalicFont=*-Italic,
BoldFont=*-SemiBold,
BoldItalicFont=*-SemiBoldItalic]{IBMPlexSansCondensed}

\setmonofont[Scale=0.91,
Extension=.otf,
UprightFont=*-Regular,
ItalicFont = IBMPlexMono-Italic.otf,
BoldFont = IBMPlexMono-Bold.otf,
BoldItalicFont = IBMPlexMono-BoldItalic.otf]{IBMPlexMono.otf}

% control de ruptura de linea
\usepackage{linebreaker}
\linebreakersetup{
	maxtolerance=90,
	maxemergencystretch=1em,
	maxcycles=4
}

% paquetes varios
\usepackage{zref-totpages}% contar el total de páginas
\usepackage{pageslts}
\usepackage{calc}
\usepackage{qrcode}% generamos el QR
\usepackage{froufrou}
\usepackage{nccfoots}
\usepackage{booktabs}
\usepackage{rotating}
\usepackage{graphicx}
\usepackage{svg}
%\usepackage[final]{pdfpages}
\usepackage[labelfont=bf,font=small,labelsep=period,format=plain]{caption}
\usepackage{ragged2e}
\usepackage{xcolor}
\usepackage[framemethod=tikz]{mdframed}
\usepackage{bchart}
\usepackage[most]{tcolorbox}
% control sobre parrafos de una linea a final de página
\usepackage{needspace}% usar donde se desea el corte \Needspace*{4\baselineskip}
% diseño de listas (opcion 1) con paralist
\usepackage{paralist}
\setdefaultenum{1)}{a)}{i)}{}
\pltopsep=0.5mm
\plitemsep=0mm
% diseño de listas (opcion 2) con enumitem
\usepackage{enumitem}
\setlist{nosep,topsep=4pt}
% rediseñamos la raya de las notas a pie aumentamos la distancia de la raya
\renewcommand{\footnoterule}{%
	\kern -3pt%
	\hrule height 0.5pt width 0.4\columnwidth%
	\kern 6pt%
	}

%% rediseño del epígrafe
%% la columna superior es tratada como párrafo y
%% la inferior marginada a la derecha
\newcommand{\epigraph}[2]{%
	\par\nobreak\noindent\par\nobreak\vspace{.5\baselineskip}
	\hfill{\small\begin{tabular}{@{}>{\raggedright\arraybackslash}m{.65\textwidth}@{}}
			#1 \\[1ex]
			\midrule
			\hfill #2
	\end{tabular}}
	\vspace{.5\baselineskip}
}

% niveles para los contadores
\setcounter{tocdepth}{0}
\setcounter{secnumdepth}{4}

% dibujo de caja contenedora, solo para desarrollo
%\usepackage{showframe}
%\renewcommand\ShowFrameLinethickness{0.1pt}
%\renewcommand\ShowFrameColor{\color{blue}}

% diseño del pie de página
\usepackage[bottom,stable,hang]{footmisc}
\setlength{\footnotesep}{10pt}% cambia el valor deseado en puntos (pt)

\makeatletter
\patchcmd\@footnotetext{\@MM}{100}{}{\fail}
\makeatother

\makeatletter
\patchcmd{\@footnotetext}{\footnotesize}{\small}{}{}% tamaño del cuerpo del texto del footnote
\makeatother
\renewcommand*{\thefootnote}{\scriptsize\sf{[\arabic{footnote}]}}% tamaño del cuerpo de puntero del footnote

% KOMA script
\usepackage{scrextend}
% \KOMAoptions{footnotes=multiple}% maybe you want to use this option?
\newcommand*\footnotemarkspace{0em} % set distance of the footnote text from the margin
\deffootnote{\footnotemarkspace}% use distance from above
{\parindent}% paragraph indent in footnotes (footnotes should never have paragraphs!)
{\makebox[\footnotemarkspace][r]{\llap{\thefootnotemark\quad}}} % footfont with period for footnote marks in footnote
%   {\makebox[\footnotemarkspace][l]{\footfont\phantom{99}\llap{\thefootnotemark.}}} % footfont with period for footnote marks in footnote

% cambiar la font specification for the name "PART"
\renewcommand\thepart{\arabic{part}}

% AJUSTE PARA UTILIZAR SUBTITULOS, CHAPTER CON ce MAYÚSCULA \Chapter{}{}
% \usepackage{relsize} %Package to set relative font size (\smaller, \larger)
\newcommand\Chapter[2]{
	\chapter[#1]{#1\\ {\fontsize{12pt}{14.4pt}\selectfont#2}}
}

\usepackage{titletoc}
% partes
\titlecontents{part}[0em]
{\addvspace{5pt}\sf\bfseries\normalsize\selectfont\filright}
{\contentslabel[\thecontentslabel]{2.5pc}}{}{}

% capitulo
\titlecontents{chapter}[1.5pc]
{\addvspace{.4em}\sf\selectfont\filright}
{\contentslabel{1.5pc}}
{\hspace*{-1.5pc}}%
{\titlerule*[1pc]{.}\contentspage[\hspace*{-4pc} {\rm\small\thecontentspage}]}%
[]

\titlecontents{section}[4.5pc]
{\small\filright}
{\contentslabel{2.5pc}}
{\hspace*{-2.5pc}}
{\titlerule*[1pc]{.}\contentspage}

% %% diseño de sección
% \titlecontents*{section}[2.5pc]
% {\small\selectfont\filright}
% {{\sffamily{\thecontentslabel}} \ }{}
% { [\textbf{\thecontentspage}]}[][\ \textbullet\ ][]

% \titlecontents*{section}[2.5pc]
% {\footnotesize\selectfont\raggedright}
% {\textbf{\thecontentslabel\adddot}\addspace}
% % {\thecontentslabel.\addspace}% despues de terminar con Tucumán usar este
% {}
% {~[\thecontentspage].\addspace}[]
% % {\addspace[\thecontentspage].\addspace}[]

% CORRIJE LA POSICIÓN DE LOS NUMEROS EN EL ÍNDICE DE FIGURAS Y CUADROS
\makeatletter
\renewcommand{\l@figure}{\@dottedtocline {1}{0}{2.5pc}}
\renewcommand{\l@table}{\@dottedtocline {1}{0}{2.5pc}}
\makeatother

\usepackage[sf,bf,compact,topmarks,calcwidth,pagestyles,clearempty,newlinetospace]{titlesec}
%diseño de parte
\makeatletter
\def\@part[#1]#2{%
	\ifnum \c@secnumdepth >-2\relax
	\refstepcounter{part}%
	\addcontentsline{toc}{part}{Parte \thepart\hspace{1em}#1}%
	\else
	\addcontentsline{toc}{part}{#1}%
	\fi
	\markboth{}{}%
	{\centering
		\interlinepenalty \@M
		\ifnum \c@secnumdepth >-2\relax
		\sf\LARGE\selectfont \partname~\thepart
		\par\vskip 20\p@%
		\fi
		\sf\Large\selectfont
		#2\par}%
	\@endpart}

\def\@spart#1{%
	\markboth{}{}%
	{\centering
		\interlinepenalty \@M
		\sf\LARGE\selectfont
		#1\par}%
	\@endpart}
\makeatother

%diseño de capitulo
\titleformat{\chapter}[display]
% {\sf \LARGE}
% {\filleft {\chaptertitlename} \thechapter}
% {2ex}{\filright}[]
{\Large}
{\vspace{-2.5cm}\centering{\textsc{\MakeLowercase\chaptertitlename}}~\thechapter}
{1.5cm}
{\filright\sf\LARGE}
[]
% \titlespacing{\chapter}{0pt}{-9ex}{0pt}

%diseño de seccion
\titleformat{\section}[hang]
{\sf\bfseries\large\raggedright}
{\thesection}{.5em}{}[]
\titlespacing{\section}
{\parindent}{18pt plus 0.5pt minus 0.5pt}{6.75pt}

%diseño de subseccion
\titleformat{\subsection}[hang]
{\sf\large\raggedright}
{\thesubsection}{.5em}{}[]
\titlespacing{\subsection}
{\parindent}{18pt plus 0.5pt minus 0.5pt}{6.75pt}

%diseño de subsubseccion
\titleformat{\subsubsection}[hang]
{\rm\bfseries\normalsize\raggedright}
{\thesubsubsection}{.5em}{}[]
\titlespacing{\subsubsection}
{\parindent}{18pt plus 0.5pt minus 0.5pt}{6.75pt}

%diseño de parrafo para usar con los autores de compilaciones
\titleformat{\paragraph}[runin]
  {\bfseries}
  {}{0em}{}
  [\mbox{ --- }]
\titlespacing{\paragraph}
  {0pt}% antes de la raya
  {6.75pt plus 0.5pt minus 0.5pt}% antes del párrafo
  {0pt}% después de la raya

%diseño de subparrafo
\titleformat{\subparagraph}[hang]
{\sf\bfseries\large\centering}
{\thesubparagraph}{.5em}{}[]
\titlespacing{\subparagraph}
{\parindent}{18pt plus 0.5pt minus 0.5pt}{6.75pt}

% DISEÑO DE CABEZALES
\renewpagestyle{plain}[]{% \footrule
\setfoot{}{}{}}
\newpagestyle{myps}[]{%
\setfoot[][][]{}{}{}
\sethead[\sf \textbf{\usepage}][][\sf \TheAuthor]
{\sf \chaptertitle}{}{\sf \textbf{\usepage}}
}
\pagestyle{myps}

\newcommand{\TheAuthor}{}
\newcommand{\Author}[1]{\renewcommand{\TheAuthor}{#1}}

% Ajustes de viudas y huérfanas
\raggedbottom
\clubpenalty=10000
\widowpenalty=10000
% \finalhyphendemerits% evitamos el corte en la última línea del párrafo
%% CON ESTAS 2 INSTRUCCIONES NUNCA VA A GUIONIZAR PALABRAS MENORES A 7 CARACTERES
% \lefthyphenmin3 %% determina el minimo de caracteres de final de linea a la izquierda
% \righthyphenmin3 %% determina el minimo de caracteres de final de linea a la derecha

% NUEVO TIPO DE ENTORNO FLOTANTE PARA FOTOGRAFIAS (\listofimagen)
\usepackage{newfloat}
\DeclareFloatingEnvironment[
fileext=lop,
listname={Índice de imágenes},
name=Imagen,
placement=ht,
%within=section,% activate it if you want
%chapterlistsgaps=on,% meaningful only if chapters exist
]{imagen}

% DISEÑO DE RAYA DEL MEDIO
\makeatletter
\def\thinskip{\hskip 0.16667em\relax}
\def\endash{--}
\def\emdash{\endash-}
\def\d@sh#1#2{\unskip#1\thinskip#2\thinskip\ignorespaces}
\def\dash{\d@sh\nobreak\endash}
\def\Dash{\d@sh\nobreak\emdash}
\def\ldash{\d@sh\empty{\hbox{\endash}\nobreak}}
\def\rdash{\d@sh\nobreak\endash}
\def\Ldash{\d@sh\empty{\hbox{\emdash}\nobreak}}
\def\Rdash{\d@sh\nobreak\emdash}
\def\hyph{-\penalty\z@\hskip\z@skip}
\def\slash{/\penalty\z@\hskip\z@skip}
\makeatother

% CENTRADO DEL AUTOR DE LOS CAPITULOS
\newcommand\nombreautor[1]{\textsc{\MakeLowercase{#1}}}

% CITA CON CAMBIO DE TAMAÑO TIPOGRAFICO
\renewenvironment{quote}
  {\normalsize\list{}{\sf\leftmargin=14pt \rightmargin=0pt}%
   \item\relax}
  {\endlist}

% elegimos el estandar para las referencias
\def\estandar{veronaAY}
%\def\estandar{veronaFN}
%\def\estandar{apa}
%\def\estandar{chicago}
\input{./files/biblatex-\estandar-config.tex}

% generamos los índices
\usepackage[xindy]{imakeidx}
\makeindex
\makeindex[name=names,title={Índice de autoras y autores del aparato bibliográfico}]
\makeindex[name=conceptos,title={Índice de conceptos}]
\makeindex[name=onomastico,title={Índice onomástico}]

\usepackage{esindex}
\DeclareIndexNameFormat{default}{%
	\usebibmacro{index:name}{\esindex[names]}
	{\namepartfamily}
	{\namepartgiven}
	{\namepartprefix}
	{\namepartsuffix}}
	\renewbibmacro*{citeindex}{%
	\ifciteindex
	{\indexnames{labelname}}
	{}}

% generamos los glosarios
\usepackage[acronym,sanitizesort,toc=false]{glossaries}
\preto\chapter{\glsresetall}
\makenoidxglossaries
\renewcommand{\glsnamefont}[1]{\sf\textbf{\textup{#1}}}
% deshabilitamos mostrar el número de página
\renewcommand{\glossaryentrynumbers}[1]{\ (Véase pág.~#1.)}
\renewcommand{\delimN}{, }
\renewcommand{\delimR}{--}

\usepackage{url}%[allowmove]
\Urlmuskip = 0mu plus 1mu
\def\UrlBreaks{\do\a\do\b\do\c\do\d\do\e\do\f\do\g\do\h\do\i\do\j\do\k\do\l\do\m\do\n\do\o\do\p\do\q\do\r\do\s\do\t\do\u\do\v\do\w\do\x\do\y\do\z\do\A\do\B\do\C\do\D\do\E\do\F\do\G\do\H\do\I\do\J\do\K\do\L\do\M\do\N\do\O\do\P\do\Q\do\R\do\S\do\T\do\U\do\V\do\W\do\X\do\Y\do\Z\do0\do1\do2\do3\do4\do5\do6\do7\do8\do9\do=\do.\do:\do\%\do?\do_\do-\do+\do/\do\#\do~}
\def\UrlFont{\rm}

% configuración de los valores decimales de los cuadros
\usepackage{siunitx}
\sisetup{output-decimal-marker={.},
	group-separator={\hspace{0.15em}},
	group-minimum-digits=3,
	table-text-alignment=center,
	detect-all,
	per-mode=fraction}

% Comando para insertar una página en blanco
\newcommand{\PaginaEnBlanco}{
    \newpage
    \thispagestyle{empty}
    {\textcolor{white}{.}} % Punto invisible para forzar la página
}

% declaramos los condicionales
\newif\ifPDF
\newif\ifBNPDF
\newif\ifPNGEPUB
\newif\ifHTMLEPUB
\newif\ifHTML
