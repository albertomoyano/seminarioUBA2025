\documentclass[oneside,onecolumn,openany,final]{book}
% Idiomas
\usepackage[french,portuguese,italian,english,german,spanish,es-ucroman,es-noshorthands]{babel}
\usepackage{csquotes}
\renewcommand{\spanishcontentsname}{Sumario}
\renewcommand{\spanishindexname}{Índice alfabético}

\usepackage{fontspec}

% configuración de los valores decimales de los cuadros
\usepackage{siunitx}
\sisetup{output-decimal-marker={.},
         group-separator={\hspace{0.15em}},
         group-minimum-digits=3,
         table-text-alignment=center,
         detect-all,
         per-mode=fraction}

% paquetes varios
\usepackage{froufrou}
\usepackage{nccfoots}
\usepackage{booktabs}
\usepackage{rotating}
\usepackage{graphicx}
\usepackage{svg}
\usepackage[final]{pdfpages}
\usepackage[labelfont=bf,font=small,labelsep=period,format=plain]{caption}
\usepackage{ragged2e}
\usepackage{xcolor}
\usepackage{bchart}
\usepackage[most]{tcolorbox}
% control sobre parrafos de una linea a final de página
\usepackage{needspace}% usar donde se desea el corte \Needspace*{4\baselineskip}
% diseño de listas (opcion 1) con paralist
\usepackage{paralist}
\setdefaultenum{1)}{a)}{i)}{}
\pltopsep=0.5mm
\plitemsep=0mm
% diseño de listas (opcion 2) con enumitem
\usepackage{enumitem}
\setlist{nosep,topsep=4pt}
% rediseñamos la raya de las notas a pie aumentamos la distancia de la raya
\renewcommand{\footnoterule}{%
	\kern -3pt%
	\hrule height 0.5pt width 0.4\columnwidth%
	\kern 6pt%
	}

%% rediseño del epígrafe
%% la columna superior es tratada como párrafo y
%% la inferior marginada a la derecha
\newcommand{\epigraph}[2]{%
	\par\nobreak\noindent\par\nobreak\vspace{.5\baselineskip}
	\hfill{\small\begin{tabular}{@{}>{\raggedright\arraybackslash}m{.65\textwidth}@{}}
			#1 \\[1ex]
			\midrule
			\hfill #2
	\end{tabular}}
	\vspace{.5\baselineskip}
}

% niveles para los contadores
\setcounter{tocdepth}{0}
\setcounter{secnumdepth}{4}

% cambiar la font specification for the name "PART"
\renewcommand\thepart{\arabic{part}}

% AJUSTE PARA UTILIZAR SUBTITULOS, CHAPTER CON ce MAYÚSCULA \Chapter{}{}
% \usepackage{relsize} %Package to set relative font size (\smaller, \larger)
\newcommand\Chapter[2]{
	\chapter[#1]{#1\\ {\fontsize{12pt}{14.4pt}\selectfont#2}}
}

\usepackage{titletoc}
% partes
\titlecontents{part}[0em]
{\addvspace{5pt}\sf\bfseries\normalsize\selectfont\filright}
{\contentslabel[\thecontentslabel]{2.5pc}}{}{}

% capitulo
\titlecontents{chapter}[2pc]
{\addvspace{.4em}\sf\selectfont\filright}
{\contentslabel{2pc}}
{\hspace*{-2pc}}%
{\titlerule*[1pc]{.}\contentspage[\hspace*{-3pc} {\rm\small\thecontentspage}]}%
[]

\titlecontents{section}[4.5pc]
{\small\filright}
{\contentslabel{2.5pc}}
{\hspace*{-2.5pc}}
{\titlerule*[1pc]{.}\contentspage}

% %% diseño de sección
% \titlecontents*{section}[2.5pc]
% {\small\selectfont\filright}
% {{\sffamily{\thecontentslabel}} \ }{}
% { [\textbf{\thecontentspage}]}[][\ \textbullet\ ][]

% \titlecontents*{section}[2.5pc]
% {\footnotesize\selectfont\raggedright}
% {\textbf{\thecontentslabel\adddot}\addspace}
% % {\thecontentslabel.\addspace}% despues de terminar con Tucumán usar este
% {}
% {~[\thecontentspage].\addspace}[]
% % {\addspace[\thecontentspage].\addspace}[]

% CORRIJE LA POSICIÓN DE LOS NUMEROS EN EL ÍNDICE DE FIGURAS Y CUADROS
\makeatletter
\renewcommand{\l@figure}{\@dottedtocline {1}{0}{2.5pc}}
\renewcommand{\l@table}{\@dottedtocline {1}{0}{2.5pc}}
\makeatother

\usepackage[sf,bf,compact,topmarks,calcwidth,pagestyles,clearempty,newlinetospace]{titlesec}
%diseño de parte
\makeatletter
\def\@part[#1]#2{%
	\ifnum \c@secnumdepth >-2\relax
	\refstepcounter{part}%
	\addcontentsline{toc}{part}{Parte \thepart\hspace{1em}#1}%
	\else
	\addcontentsline{toc}{part}{#1}%
	\fi
	\markboth{}{}%
	{\centering
		\interlinepenalty \@M
		\ifnum \c@secnumdepth >-2\relax
		\sf\LARGE\selectfont \partname~\thepart
		\par\vskip 20\p@%
		\fi
		\sf\Large\selectfont
		#2\par}%
	\@endpart}

\def\@spart#1{%
	\markboth{}{}%
	{\centering
		\interlinepenalty \@M
		\sf\LARGE\selectfont
		#1\par}%
	\@endpart}
\makeatother

%diseño de capitulo
\titleformat{\chapter}[display]
% {\sf \LARGE}
% {\filleft {\chaptertitlename} \thechapter}
% {2ex}{\filright}[]
{\Large}
{\vspace{-2.5cm}\centering{\textsc{\MakeLowercase\chaptertitlename}}~\thechapter}
{1.5cm}
{\filright\sf\LARGE}
[]
% \titlespacing{\chapter}{0pt}{-9ex}{0pt}

%diseño de seccion
\titleformat{\section}[hang]
{\sf\bfseries\large\raggedright}
{\thesection}{.5em}{}[]
\titlespacing{\section}
{\parindent}{18pt plus 0.5pt minus 0.5pt}{6.75pt}

%diseño de subseccion
\titleformat{\subsection}[hang]
{\sf\large\raggedright}
{\thesubsection}{.5em}{}[]
\titlespacing{\subsection}
{\parindent}{18pt plus 0.5pt minus 0.5pt}{6.75pt}

%diseño de subsubseccion
\titleformat{\subsubsection}[hang]
{\rm\bfseries\normalsize\raggedright}
{\thesubsubsection}{.5em}{}[]
\titlespacing{\subsubsection}
{\parindent}{18pt plus 0.5pt minus 0.5pt}{6.75pt}

%diseño de parrafo para usar con los autores de compilaciones
\titleformat{\paragraph}[runin]
  {\bfseries}
  {}{0em}{}
  [\mbox{ --- }]
\titlespacing{\paragraph}
  {0pt}% antes de la raya
  {6.75pt plus 0.5pt minus 0.5pt}% antes del párrafo
  {0pt}% después de la raya

%diseño de subparrafo
\titleformat{\subparagraph}[hang]
{\sf\bfseries\large\centering}
{\thesubparagraph}{.5em}{}[]
\titlespacing{\subparagraph}
{\parindent}{18pt plus 0.5pt minus 0.5pt}{6.75pt}

% DISEÑO DE CABEZALES
\renewpagestyle{plain}[]{% \footrule
\setfoot{}{}{}}
\newpagestyle{myps}[]{%
\setfoot[][][]{}{}{}
\sethead[\sf \textbf{\usepage}][][\sf \TheAuthor]
{\sf \chaptertitle}{}{\sf \textbf{\usepage}}
}
\pagestyle{myps}

\newcommand{\TheAuthor}{}
\newcommand{\Author}[1]{\renewcommand{\TheAuthor}{#1}}

% NUEVO TIPO DE ENTORNO FLOTANTE PARA FOTOGRAFIAS (\listofimagen)
\usepackage{newfloat}
\DeclareFloatingEnvironment[
fileext=lop,
listname={Índice de imágenes},
name=Imagen,
placement=ht,
%within=section,% activate it if you want
%chapterlistsgaps=on,% meaningful only if chapters exist
]{imagen}

% CENTRADO DEL AUTOR DE LOS CAPITULOS
\newcommand\nombreautor[1]{\textsc{\MakeLowercase{#1}}}

% CITA CON CAMBIO DE TAMAÑO TIPOGRAFICO
\renewenvironment{quote}
  {\normalsize\list{}{\sf\leftmargin=14pt \rightmargin=0pt}%
   \item\relax}
  {\endlist}

% CONFIGURACIÓN PARA BIBLATEX
\usepackage[style=philosophy-modern,
% refsection=chapter,%para obtener las referencias por capítulo
sortcites=true,
lowscauthors=true,
scauthorsbib=true,
annotation=true,
backend=biber,
labeldateparts=true,
backref=true,
useprefix=true,
citereset=chapter,
indexing=cite,
relatedformat=brackets,
publocformat=loccolonpub,
volnumformat=strings,
latinemph=true,
inbeforejournal=true,
shorthandintro=true,
texencoding=utf8,
bibencoding=utf8,
uniquelist=minyear]{biblatex}

%GENERAR LOS INDICES, DESACTIVAMOS LAS OPCIONES PARA TITULOS
\renewbibmacro*{bibindex}{%
  \ifbibindex
    {\indexnames{author}%
     \indexnames{editor}%
     \indexnames{translator}%
     \indexnames{commentator}}
    {}}

\renewbibmacro*{citeindex}{%
  \ifciteindex
    {\indexnames{author}%
     \indexnames{editor}%
     \indexnames{translator}%
     \indexnames{commentator}}
    {}}

% CAMBIAMOS URL PARA QUE APAREZCA LA LEYENDA EN LINEA
\makeatletter
\newrobustcmd{\mkbiblege}[1]{%
	\begingroup
	\blx@blxinit
	\blx@setsfcodes
	<#1>
	\endgroup}
\makeatother

\DeclareFieldFormat{url}{\bibstring{url}\space\mkbiblege{\url{#1}}}

% REDEFINIR EL FORMATO DE CITADO EN PÁGINA 00
% \DeclareFieldFormat{pagerefformat}{\mkbibparens{{\color{red}\mkbibemph{#1}}}}
\renewbibmacro*{pageref}{%
	\iflistundef{pageref}
	{}
	{\printtext[pagerefformat]{%
			\ifnumgreater{\value{pageref}}{1}
			{\bibstring{backrefpages}\ppspace}
			{\bibstring{backrefpage}\ppspace}%
			\printlist[pageref][-\value{listtotal}]{pageref}}}}

% DECLARO UNA ALIAS PARA TRATAR MOVIE COMO MISC
\DeclareBibliographyAlias{movie}{misc}

% PONE PUNTO Y COMA ENTRE NOMBRE DE VARIOS AUTORES
\renewcommand*{\multinamedelim}{\addsemicolon\space}

% CAMBIA LA TIPOGRAFIA DE LA BIBLIOGRAFIA
\renewcommand*{\annotationfont}{\small\sf}
\renewcommand*{\bibfont}{\small}
\setlength{\bibhang}{3\parindent}%modifico el valor por default (4) de indentacion del año

\defcounter{biburlnumpenalty}{3000}
\defcounter{biburlucpenalty}{6000}
\defcounter{biburllcpenalty}{9000}

%% REDISEÑO PARA OBTENER MODO LARGO EN ALGUNAS ENTRADAS
\NewBibliographyString{organizator}
\NewBibliographyString{organizators}
\NewBibliographyString{byorganizator}
\NewBibliographyString{cordinator}
\NewBibliographyString{cordinators}
\NewBibliographyString{bycordinator}
\NewBibliographyString{direction}
\NewBibliographyString{directions}
\NewBibliographyString{bydirection}
\NewBibliographyString{maestriathesis}
\NewBibliographyString{ingenieriathesis}
\NewBibliographyString{gradothesis}
\NewBibliographyString{licenciaturathesis}
\NewBibliographyString{origpubbare}
\NewBibliographyString{especialthesis}
\NewBibliographyString{documentjob}
\NewBibliographyString{posgradothesis}
\NewBibliographyString{magisterthesis}
\NewBibliographyString{dirigida}% MODIFICACIONES PARA FILMES
\NewBibliographyString{dirigidas}
\NewBibliographyString{bydirigida}
\NewBibliographyString{escrita}
\NewBibliographyString{escritas}
\NewBibliographyString{byescrita}
\NewBibliographyString{elenco}
\NewBibliographyString{elencos}
\NewBibliographyString{byelenco}
\NewBibliographyString{tifthesis}
\NewBibliographyString{mimeograph}
\NewBibliographyString{photocopy}
\NewBibliographyString{digitalprinting}

\DefineBibliographyStrings{spanish}{%
	inpreparation    = {en preparación},
	submitted        = {enviado},
	forthcoming      = {de próxima aparición},
	inpress          = {en imprenta},
	prepublished     = {previamente publicado},
	digitalprinting  = {impresión digital},
	photocopy        = {fotocopia},
	mimeograph       = {mimeo},
	dirigida         = {dirección},
	dirigidas        = {dirección},
	bydirigida       = {Dirección de},
	escrita          = {escrita},
	escritas         = {escrita},
	byescrita        = {escrita por},
	elenco           = {actúan:},
	elencos          = {actúan:},
	byelenco         = {actúan:},
	part             = {tomo},
	january          = {enero},
	february         = {febrero},
	march            = {marzo},
	april            = {abril},
	may              = {mayo},
	june             = {junio},
	july             = {julio},
	august           = {agosto},
	september        = {septiembre},
	october          = {octubre},
	november         = {noviembre},
	december         = {diciembre},
	see              = {véase},
	seealso          = {véase también},
	backrefpage      = {re\-fe\-ren\-cia ci\-ta\-da en pági\-na},
	backrefpages     = {re\-fe\-ren\-cia ci\-ta\-da en pági\-nas},
	seenote          = {véase nota},
	quotedin         = {citado en},
	idem             = {ídem},
	idemsf           = {ídem},
	idemsm           = {ídem},
	idemsn           = {ídem},
	idempf           = {ídem},
	idempm           = {ídem},
	idempn           = {ídem},
	idempp           = {ídem},
	ibidem           = {\emph{ibidem}},
	prepublished     = {previamente publicado},
	nodate           = {sin fecha},
	withcommentator  = {con comentario de},
	withannotator    = {con notas de},
	withintroduction = {con introdución de},
	withforeword     = {con prólogo de},
	withafterword    = {con epílogo de},
	bycollaborator   = {con colaboración de},
	andothers        = {\emph{et al\adddot}},
	organizator      = {org\adddot},
	organizators     = {orgs\adddot},
	byorganizator    = {org\adddot\addspace por},
	cordinator       = {coord\adddot},
	cordinators      = {coords\adddot},
	bycordinator     = {coord\adddot\addspace por},
	direction        = {dir\adddot},
	directions       = {dirs\adddot},
	bydirection      = {dir\adddot\addspace por},
	mathesis         = {Tesis de Maestría},
	ingenieriathesis = {Tesis de Ingeniería},
	magisterthesis   = {Tesis de Magíster},
	phdthesis        = {Tesis de Doctorado},
	gradothesis      = {Tesis de Grado},
	licenciaturathesis = {Tesis de Licenciatura},
	tifthesis        = {Trabajo Integrador Final},
	posgradothesis   = {Tesis de Posgrado},
	especialthesis   = {Tesis de Especialización},
	documentjob      = {documento de trabajo},
	urlseen          = {visitado el\addspace},
	translationof    = {trad\adddot\addspace de},
	translationas    = {original publicado en},
	url              = {recuperado de},
	origpubbare      = {orig\adddotspace pub\adddotspace},
	newseries        = {nueva época},
	oldseries        = {antigua época},
	byeditorfo       = {ed\adddotspace y pról\adddotspace por},
	byeditorco       = {ed\adddotspace y com\adddotspace por},
	bytranslatorfo   = {traducido \lbx@lfromlang\ y prologado por},
	pagetotal        = {págs},
	pagetotals       = {págs},
	techreport       = {informe técnico},
	resreport        = {reporte de investigación},
	file             = {archivo},
	patent           = {patente},
	patentde         = {patente alemana},
	patenteu         = {patente europea},
	patentfr         = {patente francesa},
	patentuk         = {patente británica},
	patentus         = {patente estadounidense},
	patreq           = {solicitud de patente},
	patreqde         = {solicitud de patente alemana},
	patreqeu         = {solicitud de patente europea},
	patreqfr         = {solicitud de patente francesa},
	patrequk         = {solicitud de patente británica},
	patrequs         = {solicitud de patente estadounidense},
	bycompiler       = {comp\adddotspace por},
}

% corrección para año original
\makeatletter
\renewbibmacro*{transorigstring}{%
  \iffieldundef{reprinttitle}%
  {\printtext{\ifdefstring{\bbx@origfields}{origed}
      {\bibstring{origpubbare}}%
      {\bibstring{translationas}}}\nopunct}%
  {\printtext{\bibstring{reprint}}}\nopunct}
\makeatother

% generamos los índices
\usepackage[xindy]{imakeidx}
\makeindex
\makeindex[name=names,title={Índice de autoras y autores del aparato bibliográfico}]
\makeindex[name=concepto,title={Índice de conceptos}]
\makeindex[name=onomastico,title={Índice onomástico}]

\usepackage{esindex}
\DeclareIndexNameFormat{default}{%
	\usebibmacro{index:name}{\esindex[names]}
	{\namepartfamily}
	{\namepartgiven}
	{\namepartprefix}
	{\namepartsuffix}}
\renewbibmacro*{citeindex}{%
	\ifciteindex
	{\indexnames{labelname}}
	{}}

% generamos los glosarios
\usepackage[acronym,sanitizesort,toc=true,nonumberlist]{glossaries}
\preto\chapter{\glsresetall}
\makenoidxglossaries
\renewcommand{\glsnamefont}[1]{\sf\textbf{\textup{#1}}}

% Diseño y estilo a glosario y acrónimo
\newglossarystyle{crossreflist}%
{% base it on list (adapt as required)
	\renewcommand*{\glossentry}[2]{%
		\item[\glsentryitem{##1}%
		\glstarget{##1}{\glossentryname{##1}}]
		\glossentrydesc{##1}\glspostdescription\space ##2%
		% check if the user1 key has been supplied:
		\ifglshasfield{useri}{##1}%
		{% do cross-reference
			\newline
			\glsletentryfield{\crossrefs}{##1}{useri}%
			\glsseeformat[\emph{Véase:}]{\crossrefs}{}%
		}%
		{}%
	}%
}
\setglossarystyle{crossreflist}

% deshabilitamos mostrar el número de página
%\renewcommand{\glossaryentrynumbers}[1]{Véase página #1.}
\renewcommand{\delimN}{, }
\renewcommand{\delimR}{--}

% Instrucciones de salida
%\printnoidxglossary[type=\acronymtype,title={Índice de siglas}]
%\printnoidxglossary[title={Glosario de términos}]

\usepackage{ifthen}

% fin del preambulo
